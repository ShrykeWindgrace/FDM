\documentclass{article}
\usepackage[utf8]{inputenc}
\usepackage{amsmath,amsthm,amsfonts}

% \newcommand{\lilo}{{L^\infty(\mbr;L^1(\lot))}}
% \newcommand{\lot}{L^1(M(t,\cdot,\cdot)\dx\dv)}
% \newcommand{\TT}{\mathscr T}%integration
\DeclareMathOperator{\tr}{tr}
\DeclareMathOperator{\Li}{Li}
\DeclareMathOperator{\Kn}{Kn}
\DeclareMathOperator{\sgn}{sgn}
\newcommand\ST{\rule[-1em]{0pt}{2.5em}}
\newcommand{\A}{\mathcal A}%integration
\newcommand{\binter}{( 1-d ,0]}
\newcommand{\coll}{{\iiint\limits_{\mathbb R^3\times\mathbb R^3\times\mathbb S^2}}}
\newcommand{\col}{{\iint\limits_{\mathbb R^3\times\mathbb S^2}}}
\newcommand{\dmu}{{\,\mathrm d\mu}}
\newcommand{\dom}{{\,\mathrm{d} \omega}}
\newcommand{\ds}{{\,\mathrm ds}}
\newcommand{\dt}{{\,\mathrm dt}}
\newcommand{\dw}{{\,\mathrm{d} w}}
\newcommand{\dv}{{\,\mathrm{d} v}}
\newcommand{\dx}{{\,\mathrm{d} x}}
\newcommand{\dy}{{\,\mathrm dy}}
\newcommand{\du}{{\,\mathrm du}}
\newcommand{\dr}{{\,\mathrm dr}}
\newcommand{\F}{\mathcal F}%integration
\newcommand{\G}{\mathcal G}%integration
\newcommand{\glV} {{\mathcal V}}
\newcommand{\glVo} {{\mathcal V_0}}
\newcommand{\glXo} {{\mathcal X_M(0)}}
\newcommand{\glXs} {{\mathcal X_M(s)}}
\newcommand{\glXtt} {{\bar {\mathcal X}_M(t)}}
\newcommand{\glX}[1]{{\mathcal X_M(#1)}}
\newcommand{\glXt} {{\mathcal X_M(t)}}
\newcommand{\glY}{{\mathcal Y_M}}
\newcommand{\gmi}{{g^{-\infty}}}
\newcommand{\gpi}{{g^{+\infty}}}
\newcommand{\mbo}{{\mathbb R}}
\newcommand{\mbro}{{\mathbb R}}

\newcommand{\mtt}{{\mathbb R^2}}
\newcommand{\mbr}{{\mathbb R}}
\newcommand{\mro}{{\mathbb {R}}}
\newcommand{\mbs}{{\mathbb S^2}}
\newcommand{\mst}{{\mathbb {S}^2}}
\newcommand{\mbt}{{\mathbb R^3}}
\newcommand{\mbrt}{{\mathbb R^3}}
\newcommand{\mbrd}{{\mathbb R^d}}
\newcommand{\mrt}{{\mathbb {R}^3}}
\newcommand{\mch}{{\mathcal H}}
\newcommand{\mcn}{{\mathcal {N}}}
\newcommand{\mrd}{{\mathbb R^d}}
\newcommand{\mrn}{{\mathbb {R}^n}}
\newcommand{\msd}{{\mathbb S^{d-1}}}
\newcommand{\oto}{{(\omega\otimes\omega)}}
\newcommand{\rdrd}{{\mrd\times\mrd}}
\newcommand{\Sc}{\mathcal S}%integration
\newcommand{\then}{{\Rightarrow}}
\newcommand{\T}{\mathcal T}%integration
\newcommand{\vbeta}{{\vec\beta}}
\newcommand{\ve}{{\varepsilon}}
\newcommand{\ep}{{\varepsilon}}
\newcommand{\vrho}{{\vec\rho}}
\newcommand{\vx}{{\vec x }}
\newcommand{\vy}{{\vec y }}

\newtheorem{corollary}{Corollary}
\newtheorem{definition}{Definition}
\newtheorem{lemma}{Lemma}
\newtheorem{proposition}{Proposition}
\newtheorem{theorem}{Theorem}
\newtheorem*{theorem*}{Theorem}
\newtheorem*{lemma*}{Lemma}
\renewcommand{\L}{\mathcal L}

\newcommand{\der}[1]{\frac{\mathrm d}{\mathrm {d {#1}}}}
% \numberwithin{equation}{section}
% \renewcommand{\theequation}{\thechapter.\thesection.\arabic{equation}}
% \numberwithin{theorem}{chapter}
% \renewcommand{\thetheorem}{\Roman{theorem}} 

% \newcommand{\MyPathSc}{../Scattering/Code}
% \newcommand{\MyPathCNSE}{../FD/CSNE/Final/Code}

\begin{document}

\[
	q_0(x,v) = a|x|^2 + c|v|^2 - 2 b x\cdot v - 2Bx\cdot v,
\]
\[
	q_1(x,v) = |x|^2 + |v|^2 - 2b_1 x\cdot v - 2B_1x\cdot v,
\]
\[
	q_1(-x,v)=q_1(x,-v)\ne q_1(x,v)=q_1(-x,-v),
\]
\[
	a>0,\quad c>0,\quad b\in\mbo,\quad B=-B^T,
\]
\[
	b_1 =\frac{b}{\sqrt{ac}}, \quad  B_1=\frac{1}{\sqrt{ac}}B,
\]
\[
	Q=(ac-b^2)I+B^2,\quad Q=Q^*>0,	
\]
\[
	q(t,x,v) = q_0(x-tv-(x_0-tv_0),v-v_0),
\]
\[
	F(t,x,v)=\frac{1}{1+\exp\left(\mu+\frac{1}{2}q(t,x,v)\right)}
\]
\[
	=\frac{1}{1+\exp\left(\mu+\frac{1}{2}q_0(x-tv-(x_0-tv_0),v-v_0)\right)}
\]

\[
	M_{0} = \int_{\rdrd}^{}M(t,x,v)\dx\dv
\]

\[
	= \int_{\rdrd}^{}\frac{\dx\dv}{1+\exp\left(\mu+\frac{1}{2}q_0(x-tv-(x_0-tv_0),v-v_0)\right)}.
\]
\[
	=\int_{\rdrd}^{}\frac{\dx\dv}{1+\exp\left(\mu+\frac{1}{2}q_0(x,v)\right)}.
\]

\[
	=(ac)^{3/2}\int_{\rdrd}^{}\frac{\dx\dv}{1+\exp\left(\mu+\frac{1}{2}q_1(x,v)\right)}.
\]
\[
	m_0 = \int_{\rdrd}^{}\frac{\dx\dv}{1+\exp\left(\mu+\frac{1}{2}q_1(x,v)\right)}.
\]
\[
	M_0 = (ac)^{3/2}m_0.
\]

\[
	M_{10} = \int_{\rdrd}^{}F(t,x,v)x\dx\dv
\]
\[
	= \int_{\rdrd}^{}\frac{x\dx\dv}{1+\exp\left(\mu+\frac{1}{2}q_0(x-tv-(x_0-tv_0),v-v_0)\right)}.
\]
\[
	=\int_{\rdrd}^{}\frac{(x+x_0+tv)\dx\dv}{1+\exp\left(\mu+\frac{1}{2}q_0(x,v)\right)}
\]
\[
	=\int_{\rdrd}^{}\frac{x_0\dx\dv}{1+\exp\left(\mu+\frac{1}{2}q_0(x,v)\right)}
\]
\[
	=(ac)^{3/2}m_0x_0
\]
\[
	M_{01} = \int_{\rdrd}^{}F(t,x,v)v\dx\dv= (ac)^{3/2}m_0v_0.
\]
\[
	M_{20} = \int_{\rdrd}^{}F(t,x,v)|x|^2\dx\dv
%
\]
\[
	= \int_{\rdrd}^{}\frac{|x|^2\dx\dv}{1+\exp\left(\mu+\frac{1}{2}q_0(x-tv-(x_0-tv_0),v-v_0)\right)}.
\]
\[
	=\int_{\rdrd}^{}\frac{|x+x_0+tv|^2\dx\dv}{1+\exp\left(\mu+\frac{1}{2}q_0(x,v)\right)}
\]
\[
	=\int_{\rdrd}^{}\frac{|x|^2+|x_0|^2+|tv|^2+ 2tx\cdot v\dx\dv}{1+\exp\left(\mu+\frac{1}{2}q_0(x,v)\right)}
\]
\[
	=a^{5/3}c^{3/2}m_{20}+a^{3/2}c^{5/2}t^2m_{02}+(ac)^{3/2}m_{0}|x_0|^2 +2 a^2c^2tm_{1\cdot1},
\]

\[
	m_{20} = \int_{\rdrd}^{}\frac{|x|^2\dx\dv}{1+\exp\left(\mu+\frac{1}{2}q_0(x,v)\right)}
\]
\[
	m_{02} = \int_{\rdrd}^{}\frac{|v|^2\dx\dv}{1+\exp\left(\mu+\frac{1}{2}q_0(x,v)\right)}
\]
\[
	m_{1\cdot1} = \int_{\rdrd}^{}\frac{x\cdot v\dx\dv}{1+\exp\left(\mu+\frac{1}{2}q_0(x,v)\right)}
\]
\[
	M_{1\cdot 1}= \int_{\rdrd}^{}F(t,x,v)x\cdot v\dx\dv 
\]
\[
	=\int_{\rdrd}^{}\frac{x\cdot v\dx\dv}{1+\exp\left(\mu+\frac{1}{2}q_0(x-tv-(x_0-tv_0),v-v_0)\right)}
\]
\[
	=\int_{\rdrd}^{}\frac{x\cdot (v+v_0)\dx\dv}{1+\exp\left(\mu+\frac{1}{2}q_0(x-tv-x_0,v)\right)}
\]
\[
	=\int_{\rdrd}^{}\frac{(x+x_0+tv)\cdot (v+v_0)\dx\dv}{1+\exp\left(\mu+\frac{1}{2}q_0(x,v)\right)}
\]
\[
	=a^2c^2m_{1\cdot 1}+M_0 x_0\cdot v_0 + t a^{3/2}c^{5/2}m_{02}.
\]
\[
	M_{1\times 1} =\int_{\rdrd}^{}F(t,x,v)x\times v\dx\dv 
\]
\[
	=\int_{\rdrd}^{}\frac{x\times v\dx\dv}{1+\exp\left(\mu+\frac{1}{2}q_0(x-tv-(x_0-tv_0),v-v_0)\right)}
\]
\[
	=\int_{\rdrd}^{}\frac{x\times (v+v_0)\dx\dv}{1+\exp\left(\mu+\frac{1}{2}q_0(x-tv-x_0,v)\right)}
\]
\[
	=\int_{\rdrd}^{}\frac{(x+x_0+tv)\times (v+v_0)\dx\dv}{1+\exp\left(\mu+\frac{1}{2}q_0(x,v)\right)}
\]
\[
	=a^2c^2m_{1\times 1}+M_0 x_0\times v_0.
\]
\[
	M_{02}=\int_{\rdrd}^{}F(t,x,v)|v|^2\dx\dv
\]
\[
	=\int_{\rdrd}^{}\frac{|v|^2\dx\dv}{1+\exp\left(\mu+\frac{1}{2}q_0(x-tv-(x_0-tv_0),v-v_0)\right)}
\]
\[
	=\int_{\rdrd}^{}\frac{|v+v_0|^2\dx\dv}{1+\exp\left(\mu+\frac{1}{2}q_0(x-tv-x_0,v)\right)}
\]
\[
	=\int_{\rdrd}^{}\frac{|v+v_0|^2\dx\dv}{1+\exp\left(\mu+\frac{1}{2}q_0(x,v)\right)}
\]
\[
	=M_0|v_0|^2 + a^{3/2}c^{5/2}m_{02}	
\]
\begin{align}
	M_0&=& (ac)^{3/2}m_0\\
	M_{10}&=& (ac)^{3/2}m_0x_0\\
	M_{01}&=& (ac)^{3/2}m_0v_0\\
	M_{20}&=& a^{5/3}c^{3/2}m_{20}+a^{3/2}c^{5/2}t^2m_{02}+(ac)^{3/2}m_{0}|x_0|^2+ 2a^2c^2tm_{1\cdot1}\\
	M_{02}&=& (ac)^{3/2}m_0|v_0|^2 + a^{3/2}c^{5/2}m_{02}\\
	M_{1\times 1}&=& a^2c^2m_{1\times 1}+(ac)^{3/2}m_0 x_0\times v_0\\
	M_{1\cdot 1}&=& a^2c^2m_{1\cdot 1}+M_0 x_0\cdot v_0 + t a^{3/2}c^{5/2}m_{02}
\end{align}

Since the first three lines give us $x_0$ and $v_0$ in an evident manner, we can reduce our system to
\begin{align}
	M'_0&=& (ac)^{3/2}m_0\\
%	M'_{10}&=& (ac)^{3/2}m_0x_0\\
%	M'_{01}&=& (ac)^{3/2}m_0v_0\\
	M'_{20}-2tM'_{1\times 1}+t^2M'_{02}&=& a^{5/3}c^{3/2}m_{20}\\
	M'_{02}&=& a^{3/2}c^{5/2}m_{02}\\
	M'_{1\times 1}&=& a^2c^2m_{1\times 1}\\
	M'_{1\cdot 1}&=& a^2c^2m_{1\cdot 1}
\end{align}

\textbf{On bilinear form coefficients in the case $d=1$}

Suppose we have the map
\[
	(a,b)\to \left( \int_{\mbo}^{}e^{b-a|x|^2}\dx,\, \int_{\mbo}^{}|x|^2 e^{b-a|x|^2}\dx\right)
\]
for $a>0$, $b\in\mbo$. We will find its image and show that this map is injective.

Indeed, we can always write
\[
	\mu_0=\int_{\mbo}^{}e^{b-\frac{a}{2}|x|^2}\dx = e^b \sqrt{2\pi}a^{-1/2},
\]
\[
	\mu_2=\int_{\mbo}^{}|x|^2e^{b-\frac{a}{2}|x|^2}\dx = e^b \int_{\mbo}\frac{1}{a^{3/2}}|\sqrt{a}x|^2e^{-\frac{1}{2}|\sqrt{a}x|^2}\mathrm d (\sqrt{a}x) = e^b \sqrt{2\pi}a^{-3/2},
\]
which leads to
\[
	\mu_0/\mu_2 = a,\quad \mu_0^3/\mu_2=2\pi e^{2b},
\]
which implies the invertibility of our map.

The image of our map is essentially described by the first relation $\mu_2=\mu_0/a$; we know that the moment $\mu_0$ takes all values in $(0,+\infty)$ for a fixed parameter $a$ and the parameter $b$ varying in $\mbo$. Therefore, thanks to a strictly increasing behavior of $\mu_0$ with respect to $b$, the moment $\mu_2$ also takes all possible values in $(0,+\infty)$.

To summarize, we have the map
\[
	(0,+\infty)\times \mbo \to (0,+\infty)\times(0,+\infty),\quad (a,b)\mapsto (\mu_0,\mu_2) 
\]
with
\[
	\mu_0 = e^b \sqrt{2\pi}a^{-1/2},\quad\mu_2 = e^b \sqrt{2\pi}a^{-3/2},\quad a = \frac{\mu_0}{\mu_2},\quad b = \frac{1}{2} \ln\left( \frac{\mu_0^3}{\mu_2} \right).
\]


\textbf{On bilinear form coefficients in the case $d=2$}.
Now we consider the bilinear positive definite form in the two-dimensional case.

Let $A\in M_2(\mbo)$ be a positive definite matrix and $b\in\mbo$. We define the map
\[
	(A,b)\to\left( \int_{\mtt}^{}e^{b-\frac{1}{2}(Ax,x)}\dx, \int_{\mtt}^{}x_1^2 e^{b-\frac{1}{2}(Ax,x)}\dx,\int_{\mtt}^{}x_2^2e^{b-\frac{1}{2}(Ax,x)}\dx,\int_{\mtt}^{}x_1x_2e^{b-\frac{1}{2}(Ax,x)}\dx \right)
\]
The set of positive definite matrices is an open cone; in this case we can express algebraically the condition that the matrix $A$ is symmetric positive definite ($\tr  A>0$, $\det A>0$, $A=A^T$). However, this condition can not be easily generalised to higher dimensions.

We are capable of writing some of these integrals in terms of invariants of this matrix, since the integral $\mu_0$ can be explicitly written as
\[
	\mu_0 = e^{b}\frac{2\pi}{\sqrt{\det A}}, 
\]
and after an orthogonal change of variables the sum becomes
\[
	\mu_{20}+\mu_{02} = e^{b}\int_{\mtt}^{}(y_1^2+y_2^2)e^{-\frac{\lambda_1}{2}y_2^2-\frac{\lambda_2}{2}y_2^2}\dy
\]
with $\lambda_i$ being the eigenvalues of the matrix $A$. Thus, we can write
\[
	\mu_{20}+\mu_{02} = e^b\left( \frac{2\pi}{\sqrt{\lambda_1\lambda_2^3}}+\frac{2\pi}{\sqrt{\lambda_1^3\lambda_2}} \right) = 2\pi e^b \frac{\lambda_1+\lambda_2}{\sqrt{\lambda_1^3\lambda_2^3}} = 2\pi e^b \frac{\tr A}{(\det A)^{3/2}}. 
\]

Atm I do not see an efficient way to express the remaining integrals in terms of invariants. However, we might try to do that in terms of the coefficients of the matrix $A$ after a \textit{non-orthogonal} change of variables.

For example, if we make a change of variables $x_1\to x_1+ \frac{a_{12}}{a_{11}}x_2$, then (\textbf{check it!}) we obtain
\[
	\mu_{02} = \frac{2\pi e^b a_{11}}{(\det A)^{3/2}}.
\]
A similar change of variables yields
\[
	\mu_{20} = \frac{2\pi e^b a_{22}}{(\det A)^{3/2}}
\]
Finally, upon the same type of change of variables we can arrive to
\[
	\mu_{11}=\int_{\mtt}^{}x_1x_2e^{b-\frac{1}{2}(Ax,x)}\dx=e^b\int_{\mtt}^{}\left(x_1- \frac{a_{12}}{a_{11}}x_2\right)x_2 e^{-\frac{1}{2}(a_{11}x_1^2+(a_{22}-a_{12}^2/a_{11})x_2^2)}\dx
\]
\[
	=-\frac{a_{12}}{a_{11}}e^b\int_{\mtt}^{} x_2^2 e^{-\frac{1}{2}(a_{11}x_1^2+(a_{22}-a_{12}^2/a_{11})x_2^2)}\dx=-\frac{a_{12}}{a_{11}}e^b\frac{2\pi}{\sqrt{a_{11}}(a_{22}-a_{12}^2/a_{11})^{3/2}}
%
\]
\[
	=\frac{2\pi e^b a_{12}}{(\det A)^{3/2}}.
\]
Thus, this map becomes
\[
	(b,a_{11},a_{12},a_{22})\to\left( e^{b}\frac{2\pi}{\sqrt{\det A}}, \frac{2\pi e^b a_{11}}{(\det A)^{3/2}}, \frac{2\pi e^b a_{12}}{(\det A)^{3/2}},\frac{2\pi e^b a_{22}}{(\det A)^{3/2}} \right).
\]
Hence we can say that
\[
	\frac{\mu_0^2}{\mu_{20}\mu_{02}-\mu_{11}^2} = \det A,
\]
which immediately gives us $b$ and all coefficients $a_{ij}$. It is important to notice that the following conditions must hold:
\[
	\mu_0>0,\quad \mu_{02}>0,\quad \mu_{20}>0,\quad \mu_{20}\mu_{02}-\mu_{11}^2>0,
\]
which can be seen as the condition that the matrix \[
	\begin{pmatrix}
		\mu_{20}&\mu_{11}\\
		\mu_{11}&\mu_{02}
	\end{pmatrix}>0
\]
is symmetric positive definite (together with $\mu_0>0$, of course).

We can write the explicit formulas for the inverse map (even though we don't want to).


\textbf{Adapt these results for FD case}

The unidimensional case will require some polylogarithms, the bidimensional - no idea.

The unidimensional case stops at the requirement to have the strict convexity of the function $\mathcal G_{3/2}$, which we don't have (recall that this was shown only for $p\ge 2$ in our article).

\textbf{Remark}

Note that the desired case~--- with a very specific bilinear positive form and a selection of moments~--- does not satisfy the full conditions. Previously we considered, esentially, the integral of $xx^T$ weighted by our exponential. Here we don't have such a luxury.

\textbf{Limited 1D case}
Consider the map
\[
	T:(a,b)\mapsto (T_0,T_2),
\]
\[
	T_0 = \int_{\mbo}^{}\frac{\dx}{1+\exp(a+bx^2)},\quad T_2=\int_{\mbo}^{}\frac{x^2\dx}{1+\exp(a+bx^2)}.
\]
The gradient of this map writes
\[
	-\begin{pmatrix}
		\int_{\mbo}^{}\dy(x)&\int_{\mbo}^{}x^2\dy(x)\\\int_{\mbo}^{}x^2\dy(x)&\int_{\mbo}^{}x^4\dy(x)
	\end{pmatrix}
%
\]
with the measure $\dy(x)=\frac{\exp(a+bx^2)\dx}{(1+\exp(a+bx^2))^2}$. The Cauchy-Bunyakovskiy-Schwarz inequality assures that $|\det(\nabla T)|\ne0$, hence the map $T$ is locally invertible.

Suppose that have parameters $a_1>a_2$ and $b_1<b_2$ such that
$T_0(a_1,b_1)=T_0(a_2,b_2)$. Can we show that in this case $T_2(a_1,b_1)\ne T_2(a_2,b_2)$?

We have a critical point $x_*$ given by
\[
	x_*=\sqrt{\frac{a_1-a_2}{b_2-b_1}}.
\]
Consider the function \[
	f:\mbo\to\mbo,\quad f(x) = \frac{1}{1+\exp(a_1+b_1x^2)}-\frac{1}{1+\exp(a_2+b_2x^2)}.
%
\]
This function has the following properties:
\begin{itemize}
	\item $f(x)=0\iff |x|=x_*$;
	\item $f(x)>0\iff |x|>x_*$;
	\item $f(x)<0\iff |x|<x_*$;
	\item this function is $C^{\infty}$;
	\item $\lim_{|x|\to +\infty}f(x)=0$;
	\item $\int_{\mbo}^{}f(x)\dx=0$ by our choice of parameters $a_i$ and $b_i$.
\end{itemize}

Therefore, we can write an estimate
\[
	\int_{\mbo}^{}x^2f(x)\dx = \int_{|x|<x_*}^{}x^2f(x)\dx + \int_{|x|>x_*}^{}x^2f(x)\dx
\]\textbf{TBC\dots}

\textbf{Lemma analysis-style}

\begin{lemma}
	Let $d\in\mathbb N_*$. Let also $f$ and $g$ be positive, continuous, and strictly increasing functions defined on $[0,+\infty)$ such that the integrals 
		\[
			T_0 = \int_{\mbrd}^{}\frac{\dx}{1+\exp(a+bg(|x|))},\quad T_f = \int_{\mbrd}^{}\frac{f(|x|)\dx}{1+\exp(a+bg(|x|))}
		\]
		exist for all $a\in\mbo$ and $b>0$.

		Then the map $(a,b)\mapsto (T_0,T_f)$ is injective.
	\label{le:basic}
\end{lemma}
\begin{proof}
	Suppose that $T_0(a_1,b_1)=T_0(a_2,b_2)$ and $(a_1,b_1)\ne (a_2,b_2)$ with $a_1\le a_2$.
	If $a_1=a_2$, then for monotonicity reasons $b_1=b_2$, which leads to a contradiction.
	Suppose now that $a_1<a_2$.
	For the same reasons $b_1>b_2$, and we can therefore consider the function
	\[
		h(x) = \frac{1}{1+\exp(a_2+b_2g(|x|))}-\frac{1}{1+\exp(a_1+b_1g(|x|))}.
	\]
	Define also $r_*>0$ as a solution of the equation $a_2+b_2g(r)=a_2+b_2g(r)$. This equation has a unique (since $g$ is strictly monotone) solution, otherwise $f$ is sign-constant, which  contradicts our choice of $a_i$ and $b_i$.
	The function $f$ has the following poperties:
	\begin{itemize}
%		\item
	\item $h(x)=0\iff |x|=r_*$;
	\item $h(x)>0\iff |x|>r_*$;
	\item $h(x)<0\iff |x|<r_*$;
	\item this function is $C^{\infty}$;
	\item $\lim_{|x|\to +\infty}h(x)=0$;
	\item $\int_{\mbrd}^{}h(x)\dx=T_0(a_2,b_2)-T_0(a_1,b_1)=0$ by our choice of parameters $a_i$ and $b_i$.
	\end{itemize}
	Now we can write the following estimation %(the constant $c_d$ is a measure of unit sphere in $\mrd$):
	\[
		T_f(a_2,b_2)-T_f(a_1,b_1)=\int_{\mbrd}^{}f(|x|)h(x)\dx=\int_{|x|<r_*}^{}f(|x|)h(x)\dx+\int_{|x|\ge r_*}^{}f(|x|)h(x)\dx
	\]
	\[
		> \int_{|x|<r_*}^{}f(r_*)h(x)\dx+\int_{|x|\ge r_*}^{}f(r_*)h(x)\dx=f(r_*)\int_{\mbrd}^{}h(x)\dx=0.
	\]
	Note that these inequalities use the strict monotonicity of $f$ and the sign change of $h$ at $|x|=r_*$.

	We can therefore conclude that for this choice of $a_i$ and $b_i$ the corresponding values of $T_f$ are different, and the map $(a,b)\mapsto (T_0,T_f)$ is indeed injective.

\end{proof}
\begin{corollary}
	As a direct corollary of the lemma \ref{le:basic} one can establish the injectiveness of the map 
	$(\mu,\theta)\mapsto (\rho, \mathcal E)$ which was proven by a by far more complex technique in one of our previous works. That technique works only for $d\ge2$, the lemma \ref{le:basic} works for $d\ge1$. The functions are $f(|x|)=g(|x|)=|x|^2$.
\end{corollary}

\begin{lemma}
	Let $a\in\mbo$, $b>0$, and the functions $f_i:\mbo\to(0,+\infty)$ are strictly positive and strictly increasing. Suppose also that $0<p<q$ and the map
	\[
		M:(a,b)\to (b^pf_1(a),b^q f_2(a))
	\]
	is injective on $\mbo\times(0,+\infty)$.

	Then the function $a\to \frac{f_1^q(a)}{f_2^p(a)}$ is injective on $\mbo$.
	\label{le:inject}
\end{lemma}
\begin{proof}
	Suppose that the function $a\to \frac{f_1^q(a)}{f_2^p(a)}$ is not injective. Then there exist $a_1\ne a_2$ such that
	\[
		\frac{f_1^q(a_1)}{f_2^p(a_1)}=\frac{f_1^q(a_2)}{f_2^p(a_2)}
	\]
	or
	\[
		\frac{f_1^q(a_1)}{f_1^q(a_2)}=\frac{f_2^p(a_1)}{f_2^p(a_2)}.
	\]
	Now chose $0<b_*\ne1$ such that 
	\[
		f_1^q(a_1) = b_*^{pq}f_1^{q}(a_2).
	\]
	This choice, together with the previous equation, implies that
	\[
		f_2^p(a_1) = b_*^{pq}f_2^p(a_2),
	\]
	and, therefore, that the map $M$ is not injective, because $M(a_1,1)=M(a_2,b_*)$ and $b_*\ne 1$.


\end{proof}

One of the questions is whether this injectivity implies the mentioned convexity of those functions $\mathcal G_p$.

\begin{corollary}
	The lemma \ref{le:inject} implies strict convexity of the function $\mathcal G_p(a)$ for $p\ge 1$ with respect to $a\in \mbo$.

	Indeed, this convexity is equivalent to strict increase of the function (up to a multiplicative constant) 
	\[
		a\to\frac{\F_{p-1}(a)}{(\F_{p}(a))^{\frac{p-1}{p}}}.
	\]
	This strict increasing is equivalent to a strict increasing of a function
\[
	a\to\frac{\F^{p}_{p-1}(a)}{\F_{p}^{p-1}(a)}.
	\]
	In its turn, this strict increasing is a consequence of injectivity of the latter function and its continuity.
	The continuity is granted by the nature of polylogarithmic integrals, and the injectivity can be obtained via lemma \ref{le:inject} applied to functions
	\[
		f_1(a)=\F_{p-1}(a),\quad f_2(a)\F_{p},\quad q=p-1.
	\]
	All conditions of the lemma \ref{le:inject} are readily satisfied except for the injectivity of the map \[
		(a,b)\to(b^p\F_{p-1}(a),b^{p-1}\F_{p}(a)
	\]<++>
\end{corollary}<++>


\textbf{Let us try to generalise this result to $2$D}
We will consider the $2$D-case.We can transform them into
\[
	1,\quad  y_1^2+y_2^2,\quad  (y_1^2-y_2^2)\cos2\theta-2y_1y_2\sin 2\theta, \quad (y_1^2-y_2^2)\sin2\theta+2y_1y_2\cos2\theta
\]
Goal~--- find $a$, $\sigma_i$, and $\theta$, hope that they are unique.

Judging by the last two identities, we will obtain the system of equations
\[
	T_{(2-2)}\cos2\theta-T_{11}\sin2\theta=T_3,\quad T_{(2-2)}\sin2\theta+T_{11}\cos2\theta=T_4,
\]
which seems to have a unique solution in $[0,2\pi)$. However, we do not have the values of $T_{(2-2)}$ and $T_{11}$, so this point has, atm, no practical value.



\textbf{New approach to $2$D.}

We will consider the simplified version:
\[
	\begin{pmatrix}
		\mu_0\\
		\mu_{20}\\
		\mu_{02}
	\end{pmatrix}
	=
	\int_{\mbo\times\mbo}^{}\begin{pmatrix}
		1\\
		x_1^2\\
		x_2^2
	\end{pmatrix}
	\frac{\dx_1\dx_2}{1+\exp(-a+b_1x_1^2+b_2x_2^2)}.
\]
We will scale this vector to factor out $b_i$,, then apply the polar coordinates to obtain the invertibility result.
We write
\[
	\begin{pmatrix}
		\mu_0\\
		\mu_{20}\\
		\mu_{02}
	\end{pmatrix}
	=
	\int_{\mbo\times\mbo}^{}\begin{pmatrix}
		\frac{1}{\sqrt{b_1b_2}}\\
		\frac{1}{b_1\sqrt{b_1b_2}}u_1^2\\
		\frac{1}{b_2\sqrt{b_1b_2}}u_2^2
	\end{pmatrix}
	\frac{\du_1\du_2}{1+\exp(-a+u_1^2+u_2^2)}
\]
\[
=
\iint_{(0,+\infty)\times[0,2\pi)}\begin{pmatrix}
		\frac{1}{\sqrt{b_1b_2}}\\
		\frac{1}{b_1\sqrt{b_1b_2}}r^2\cos^2\varphi\\
		\frac{1}{b_2\sqrt{b_1b_2}}r^2\sin^2\varphi
	\end{pmatrix}
	\frac{r\,\mathrm dr\,\mathrm d\varphi}{1+\exp(-a+r^2)}
\]
\[
=
\int_{0}^{+\infty}\begin{pmatrix}
		\frac{2\pi r}{\sqrt{b_1b_2}}\\
		\frac{\pi}{b_1\sqrt{b_1b_2}}r^3\\
		\frac{\pi}{b_2\sqrt{b_1b_2}}r^3
	\end{pmatrix}
	\frac{\mathrm dr}{1+\exp(-a+r^2)}
\]
\[
=
\int_{0}^{+\infty}\begin{pmatrix}
		\frac{\pi }{\sqrt{b_1b_2}}\\
		\frac{\pi}{2b_1\sqrt{b_1b_2}}w\\
		\frac{\pi}{2b_2\sqrt{b_1b_2}}w
	\end{pmatrix}
	\frac{\mathrm dw}{1+\exp(-a+w)}
\]
\[
=
\begin{pmatrix}
		\frac{\pi }{\sqrt{b_1b_2}}\Gamma(1)\F_1(a)\\
		\frac{\pi}{2b_1\sqrt{b_1b_2}}\Gamma(2)\F_2(a)\\
		\frac{\pi}{2b_2\sqrt{b_1b_2}}\Gamma(2)\F_2(a)
	\end{pmatrix}
=
\begin{pmatrix}
		\frac{\pi }{\sqrt{b_1b_2}}\F_1(a)\\
		\frac{\pi}{2b_1\sqrt{b_1b_2}}\F_2(a)\\
		\frac{\pi}{2b_2\sqrt{b_1b_2}}\F_2(a)
	\end{pmatrix}.
\]
This chain of equalities imply that
\[
	\sqrt{2/\pi}\frac{\mu_0}{\sqrt[4]{\mu_{20}\mu_{02}}} = \frac{\F_1(a)}{\F_2^{1/2}(a)} = 2\frac{\mathrm d}{\mathrm da}\G_{2}(a),
\]
and the latter function is strictly increasing (cf. our other results).
This equality gives us, on the one hand, the sufficient condition on $\mu_{\alpha}$ to be an image of our map, and on the other hand it gives the value of $a$ implicitly.
After that we can obtain $b_i$ by a simple algebraic transformation.

I imagine that we can exploit a similar approach in order to get a glimpse on what happens when we have a generic positive definite matrix $B = \begin{pmatrix}
	b_{11}&b_{12}\\b_{12}&b_{22}
\end{pmatrix}$.

 The measure will be $\frac{1}{1+\exp(a+b_{11}x_1^2+2b_{12}x_1x_2+b_{22}x_2^2)}$ with the condition that the matrix  is positive definite. The test functions will be $1$, $x_1^2$, $x_{2}^2$, and $x_1x_2$.

 \textbf{remark from previous days:} The quadratic form $x^TBx$, since it is positive definite, defines an ellipse with major axes, say, $0<\sigma_1\le\sigma_2$, and rotation angle $\theta$. These parameters can be expressed in terms of $b_{ij}$. If, by some miracle, $\sigma_1=\sigma_2$, then we might be able to use the same technique as in the lemma \ref{le:basic}. If $\sigma_1<\sigma_2$, then that technique definitely does not work.

The measure is (after a rotation by an angle $\theta$) 
\[
	\frac{1}{1+\exp(a+\sigma_1y_1^2+\sigma_2y_2^2)}.
\]
The rotations are
\begin{align*}
	y_1 &=  x_1\cos \theta-x_2\sin\theta\\
	y_2&= x_1\sin\theta+x_2\cos\theta
	%\label{}
\end{align*}
or
\begin{align*}
	x_1 &=  y_1\cos \theta+y_2\sin\theta\\
	x_2&= y_1\sin\theta-y_2\cos\theta
	%\label{}
\end{align*}


The test functions are, therefore,
\[
1,\quad  y_1^2\cos^2\theta+y_1y_2\sin 2\theta+y_2^2\sin^2\theta,\]\[  y_1^2\sin^2\theta-y_1y_2\sin 2\theta+y_2^2\cos^2\theta, \quad \frac{1}{2}(y_1^2-y_2^2)\sin2\theta+y_1y_2\cos2\theta
\]

This being said, we obtain the following chain of identities:
\[
	\begin{pmatrix}
		\mu_0\\
		\mu_{20}\\
		\mu_{02}\\
		\mu_{11}
	\end{pmatrix}
	=
	\int_{\mbo\times\mbo}^{}\begin{pmatrix}
		\frac{1}{\sqrt{\sigma_1\sigma_2}}\\
		\frac{y_1^2\cos^2\theta}{\sigma_1\sqrt{\sigma_1\sigma_2}}+\frac{y_1y_2\sin 2\theta}{\sigma_1\sigma_2}+\frac{y_2^2\sin^2\theta}{\sigma_2\sqrt{\sigma_1\sigma_2}}\\
		\frac{y_1^2\sin^2\theta}{\sigma_1\sqrt{\sigma_1\sigma_2}}-\frac{y_1y_2\sin 2\theta}{\sigma_1\sigma_2}+\frac{y_2^2\cos^2\theta}{\sigma_2\sqrt{\sigma_1\sigma_2}}\\
		\frac{y_1^2\sin2\theta}{2\sigma_1\sqrt{\sigma_1\sigma_2}}+\frac{y_1y_2\cos 2\theta}{\sigma_1\sigma_2}-\frac{y_2^2\sin2\theta}{2\sigma_2\sqrt{\sigma_1\sigma_2}}\\
	\end{pmatrix}
	\frac{\dy_1\dy_2}{1+\exp(-a+y_1^2+y_2^2)}
\]
After passing to polar coordinates we obtain
\[
	\begin{pmatrix}
		\mu_0\\
		\mu_{20}\\
		\mu_{02}\\
		\mu_{11}
	\end{pmatrix}
	=
	\int_{\mbo\times\mbo}^{}\begin{pmatrix}
		\frac{1}{\sqrt{\sigma_1\sigma_2}}\\
	\frac{r^2\cos^2\varphi\cos^2\theta}{\sigma_1\sqrt{\sigma_1\sigma_2}}+\frac{r^2\sin2\varphi\sin 2\theta}{2\sigma_1\sigma_2}+\frac{r^2\sin^2\varphi\sin^2\theta}{\sigma_2\sqrt{\sigma_1\sigma_2}}\\
		\frac{r^2\cos^2\varphi\sin^2\theta}{\sigma_1\sqrt{\sigma_1\sigma_2}}-\frac{r^2\sin2\varphi\sin 2\theta}{2\sigma_1\sigma_2}+\frac{r^2\sin^2\varphi\cos^2\theta}{\sigma_2\sqrt{\sigma_1\sigma_2}}\\
		\frac{r^2\cos^2\varphi\sin2\theta}{2\sigma_1\sqrt{\sigma_1\sigma_2}}+\frac{r^2\sin2\varphi\cos 2\theta}{2\sigma_1\sigma_2}-\frac{r^2\sin^2\varphi\sin2\theta}{2\sigma_2\sqrt{\sigma_1\sigma_2}}\\
	\end{pmatrix}
	\frac{r\,\mathrm dr\,\mathrm d \varphi}{1+\exp(-a+r^2)},
\]
Which, upon integration with respect to $\varphi$ simplifies to
\[
	\begin{pmatrix}
		\mu_0\\
		\mu_{20}\\
		\mu_{02}\\
		\mu_{11}
	\end{pmatrix}
	=\pi
	\int_{0}^{+\infty}\begin{pmatrix}
		\frac{2}{\sqrt{\sigma_1\sigma_2}}\\
	\frac{r^2\cos^2\theta}{\sigma_1\sqrt{\sigma_1\sigma_2}}+\frac{r^2\sin^2\theta}{\sigma_2\sqrt{\sigma_1\sigma_2}}\\
		\frac{r^2\sin^2\theta}{\sigma_1\sqrt{\sigma_1\sigma_2}}+\frac{r^2\cos^2\theta}{\sigma_2\sqrt{\sigma_1\sigma_2}}\\
		\frac{r^2\sin2\theta}{2\sigma_1\sqrt{\sigma_1\sigma_2}}-\frac{r^2\sin2\theta}{2\sigma_2\sqrt{\sigma_1\sigma_2}}\\
	\end{pmatrix}
	\frac{r\,\mathrm dr}{1+\exp(-a+r^2)},
\]
and in its turn it becomes
\[
	\begin{pmatrix}
		\mu_0\\
		\mu_{20}\\
		\mu_{02}\\
		\mu_{11}
	\end{pmatrix}
	=\frac{\pi}{2}
	\begin{pmatrix}
		\frac{2\F_1(a)}{\sqrt{\sigma_1\sigma_2}}\\
	\left(\frac{\cos^2\theta}{\sigma_1\sqrt{\sigma_1\sigma_2}}+\frac{\sin^2\theta}{\sigma_2\sqrt{\sigma_1\sigma_2}}\right)\F_2(a)\\
		\left(\frac{\sin^2\theta}{\sigma_1\sqrt{\sigma_1\sigma_2}}+\frac{\cos^2\theta}{\sigma_2\sqrt{\sigma_1\sigma_2}}\right)\F_2(a)\\
		\left(\frac{\sin\theta\cos\theta}{\sigma_1\sqrt{\sigma_1\sigma_2}}-\frac{\sin\theta\cos\theta}{\sigma_2\sqrt{\sigma_1\sigma_2}}\right)\F_2(a)\\
	\end{pmatrix}
\]
\textbf{remark:} if somehow $\sigma_1=\sigma_2$, then immediately $\mu_{11}=0$ and $\mu_{20}=\mu_{02}$.

Let us try this one:

\[
	\frac{2\sqrt{\sigma_1\sigma_2}}{\pi\F_2(a)}\left(\mu_{02}+\mu_{20} + 2\mu_{11}   \right)= \frac{\cos^2\theta}{\sigma_1}+\frac{\sin^2\theta}{\sigma_2}+\frac{\cos^2\theta}{\sigma_2}+\frac{\sin^2\theta}{\sigma_1}+\frac{2\sin\theta\cos\theta}{\sigma_1}-\frac{2\sin\theta\cos\theta}{\sigma_2}
\]
\[
	=\frac{(\cos\theta+\sin\theta)^2}{\sigma_1}+\frac{(\cos\theta-\sin\theta)^2}{\sigma_2}.
\]
Similarly
\[
	\frac{2\sqrt{\sigma_1\sigma_2}}{\pi\F_2(a)}\left(\mu_{02}+\mu_{20} - 2\mu_{11}   \right)= \frac{\cos^2\theta}{\sigma_1}+\frac{\sin^2\theta}{\sigma_2}+\frac{\cos^2\theta}{\sigma_2}+\frac{\sin^2\theta}{\sigma_1}-\frac{2\sin\theta\cos\theta}{\sigma_1}+\frac{2\sin\theta\cos\theta}{\sigma_2}
\]
\[
	=\frac{(\cos\theta-\sin\theta)^2}{\sigma_1}+\frac{(\cos\theta+\sin\theta)^2}{\sigma_2}.
\]
Of course, we have
\[
	\frac{2\sqrt{\sigma_1\sigma_2}}{\pi\F_2(a)}\left(\mu_{02}+\mu_{20}    \right)=\frac{1}{\sigma_1}+\frac{1}{\sigma_2}
\]
and
\[
	\frac{2\sqrt{\sigma_1\sigma_2}}{\pi\F_2(a)}\left(\mu_{02}-\mu_{20}    \right)=\cos2\theta\left(\frac{1}{\sigma_1}-\frac{1}{\sigma_2}\right),
\]
\[
	\frac{2\sqrt{\sigma_1\sigma_2}}{\pi\F_2(a)}\mu_{11}=\frac{1}{2}\sin2\theta\left(\frac{1}{\sigma_1}-\frac{1}{\sigma_2}\right).
\]

\textbf{NB!} $\theta\in(-\pi/4,\pi/4]$.

Here we have several cases
\begin{enumerate}
	\item $\mu_{11}=0$, $\mu_{02}\ne \mu_{20}$. This implies that $\theta=0$ and therefore $b_{12}=0$; we can use the previous approach.
	\item $\mu_{11}=\mu_{02}-\mu_{20}=0$. In this case necessarily $\sigma_1=\sigma_2$, hence $\theta$ is undetermined and by our choice we put $\theta=0$ and use the previous approach.
	\item $\mu_{11}\ne0$, $\mu_{02}=\mu_{20}$. Here necessarily $\theta=\pi/4$. we can now say that\dots
	\item $\mu_{11}\ne0$, $\mu_{02}\ne\mu_{20}$. We can say that
		\[
			\tan (2\theta)=\frac{2\mu_{11}}{\mu_{02}-\mu_{20}}
		\]
		or
		\[
			\theta=\frac 12 \arctan\left( \frac{2\mu_{11}}{\mu_{02}-\mu_{20}} \right).
		\]
		Also,
		\[
			\cos (2	\theta)=\frac{1}{\sqrt{1+\tan^2(2\theta)}}=\frac{|\mu_{02}-\mu_{20}|}{\sqrt{(\mu_{02}-\mu_{20})^2+4\mu_{11}^2}},
		\]
		\[
			\sin(2\theta)=\frac{2|\mu_{11}|}{\sqrt{(\mu_{02}-\mu_{20})^2+4\mu_{11}^2}}\sgn\left( \frac{\mu_{11}}{\mu_{02}-\mu_{20}} \right),
		\]<++>
		\[
			\cos\theta = \sqrt{\frac{1+\cos(2\theta)}{2}}=
		\]<++>
\end{enumerate}<++>
\textbf{Sidestep: }
\begin{lemma}
	Let $A\in M_d(\mbo)$, $A=A^*>0$, $B=A^{-1}$, then
	\[
		\int_{\mrd}^{}x_px_q\exp\left(-\frac 12(Ax,x)\right)\dx=(2\pi)^{d/2}\sqrt{\det(A)}B_{pq}.
	\]
	\label{le:cross:classic}
\end{lemma}
\begin{proof}
	We can show this result with the help of the Fourier transform:
	\[
		\int_{\mrd}^{}x_px_q\exp\left(-\frac{1}{2}(Ax,x)\right)\dx=
		\left.-(2\pi)^{d/2}\sqrt{\det(A)}\partial_{\xi_p}\partial_{\xi_q} \exp\left(-\frac 12(B\xi,\xi)\right)\right|_{\xi=0}
	\]
	\[
		=\left.(2\pi)^{d/2}\sqrt{\det(A)}\partial_{\xi_p}\left(  \exp\left(-\frac 12(B\xi,\xi)\right)\sum_{j}B_{qj}\xi_j\right)\right|_{\xi=0}
	\]
\[
		=\left.(2\pi)^{d/2}\sqrt{\det(A)}\left(  \exp\left(-\frac 12(B\xi,\xi)\right)B_{qp}-\exp\left(-\frac 12(B\xi,\xi)\right)\left( \sum_{j}B_{pj}\xi_j \right)\left( \sum_{j}B_{qj}\xi_j \right)\right)\right|_{\xi=0}
\]
\[
	=(2\pi)^{d/2}\sqrt{\det(A)}B_{qp},
\]
which yields the necessary result.
\end{proof}
\textbf{Generalisation:} I hope for a generalised version of this lemma, something along the lines
\begin{lemma}
	Let $A\in M_d(\mbo)$, $A=A^*>0$ and suppose that $f:[0,+\infty)\to [0,+\infty)$ is a smooth, rapidly decaying function. Suppose also that
	\[
		\int_{\mrd}^{}x_1x_2f((Ax,x))\dx=0,
	\]
	then $ (A^{-1})_{12}=0$.
	\label{le:cross:hypothesis}
\end{lemma}
Or at least for the case $f_a(r)=\frac{1}{1+\exp(r-a)}$.
Coarea (from wiki):
\[
\int_\Omega g(x) |\nabla u(x)|\, dx = \int_{-\infty}^\infty \left(\int_{u^{-1}(t)}g(x)\,dH_{n-1}(x)\right)\,dt
\]
Whose modulus of gradient is the function $f$?

We know that there exist an orthogonal matrix $R$ and positive diagonal matrix $D$ such that
\[
	R^*D^*ADR=I,
\]
we make a change of variables $x=RDy$, $\dx=\det(D)\dy$, and the test matrix $T=\frac 12(e_{12}+e_{21})$ becomes
\[
	R^*D^*TDR = \frac 12 d_1d_2 D^*R^* T RD,
\]
thus 
\[
	\int_{\mrd}(TRDy,RDy)f(|y|^2)\dy=0.
\]
Let $F$ be smooth such that $|\nabla_y F(|y|)|=f(|y|^2)$ (\textit{does it exist?}), then
\[
	\int_{0}^{+\infty}\dt\left( \int_{F^{-1}(t)}(TRDy,RDy)\,\mathrm d H_{d-1}(y) \right)=0.
\]
Level sets of $F$ are spheres; we can therefore say that
\[
	\int_{0}^{+\infty}t^{d+1}\chi_{t\in im(F)}\left( \int_{|y|=1}(TRDy,RDy)\,\mathrm d H_{d-1}(y) \right)\dt=0.
\]
In its turn this implies that
\[
	\int_{|y|=1}(TRDy,RDy)\,\mathrm d H_{d-1}(y)=0.
\]
What can pull from it? Virtually nothing, to be honest.

We have a symmetric matrix $P$ with eigenvalues $0$ (algebraic multiplicity $d-2$) and two other eigenvalues such that
\[
	\int_{\msd}(Px,x)\dx=0.
\]
Is there anything we can tell about it? We do! Its other eigenvalues are $\pm\lambda $, and nothing beyond that.

\textbf{After sidestep}

This lemma will give us a way out for fully-generated matrices:
\[
	M_2=\int_{\mrd}^{}xx^T\exp\left( a-\frac{1}{2}(Ax,x) \right)\dx = e^a(2\pi)^{d/2}\sqrt{\det(A)}A^{-1},
\]
\[
	\mu_0=\int_{\mrd}\exp\left(a -\frac{1}{2}(Ax,x) \right)\dx =e^a (2\pi)^{d/2}\sqrt{\det(A)},
\]
hence
\[
	\left(  \frac{1}{\mu_0}M_2\right)^{-1}=A, \quad a = \ln\left( \frac{\mu_0}{(2\pi)^{d/2}\sqrt{\det(A)}} \right).
\]

\begin{lemma}
	Let $f:(0,+\infty)\to(0,+\infty) $ and $A\in M_d(\mbo)$, $A=A^T>0$ such that
	\[
		\int_{0}^{\infty}(1+r^2)r^{d-1}f(r^2)\dr<+\infty,
	\]
	then
	\[
		\int_{\mrd}xx^Tf(x^TAx)\dx = \frac{|\msd|}{d\sqrt {\det(A)}} \int_{0}^{\infty}r^{d-1}f(r^2)\dr  ,
	\]
	\[
		\int_{\mrd}f(x^TAx)\dx = \frac{|\msd|}{\sqrt {\det(A)}}  \int_{0}^{\infty}r^{d+1}f(r^2)\dr A^{-1}.
	\]
\end{lemma}
\begin{proof}
A simple change of variables $y=A^{1/2}x$ is sufficient:
\[
	x=A^{-1/2}y,\quad \dx =\frac{1}{\sqrt {\det(A)}}\dy,
\]
hence
	\[
		\int_{\mrd}f(x^TAx)\dx 
		=\frac{1}{\sqrt {\det(A)}}\int_{\mrd}f(y^T y)\dy
		=\frac{|\msd|}{\sqrt {\det(A)}}\int_{\mrd}r^{d-1}f(r^2)\dr
	\]
and
\[
	\int_{\mrd}x x^Tf(x^TAx)\dx  
	=\frac{1}{\sqrt {\det(A)}} \int_{\mrd}A^{-1/2}yy^TA^{-1/2}f(y^T y)\dy
\]
\[
	=\frac{|\msd|}{d\sqrt {\det(A)}} A^{-1/2} \int_{0}^{+\infty}r^{d+1} I f(r^2)\dr A^{-1/2}
\]
\[
	=\frac{|\msd|}{d\sqrt {\det(A)}}\int_{0}^{+\infty}r^{d+1}   f(r^2)\dr  A^{-1}  .
\]
\end{proof}


Consider the problem
\[
    A_i=A_i^T\in M_d(\mbo), i=1,\dots,k
\]
such that the family $\{A_i\}_{i\le k}$ is linearly independent.

Take now   a linear combination
\[
    Q = \sum_{i=1}^k \alpha_i A_i>0.
\]
We know only the test values (moments of some sort) 
\[
    \mu_i = \frac{A_i:Q^{-1}}{\sqrt{\det Q}}.
\]
We have two questions:
\begin{itemize}
    \item what is the image of the map $\vec \alpha\mapsto \vec \mu$?
    \item is the map $\vec \alpha\mapsto \vec \mu$ injective?
\end{itemize}
It is quite easy to answer these question when, for example, for $d=2$ and $k=1$ (and can be even extended to the case $\det Q\ne 0$).

Indeed, in this case we have
\[
    Q= \alpha A, \quad Q^{-1} = \frac{1}{\alpha}A^{-1},\quad \det Q = \alpha^2\det A, \quad A:Q^{-1}=2,
\]
so
\[
    \mu = \frac{A:Q^{-1}}{\sqrt{\det Q}}=\frac{2}{\alpha\sqrt{\alpha^2\det A }}.
\]
Note that necessarily $\det A>0$ (the matrix $A$ is either positive definite or negative definite), therefore we can say that
\[
    \alpha |\alpha| =  \frac{2}{\mu\sqrt{\det A}},
\]
and
\[
    \alpha   = \sqrt{\frac{2}{|\mu|\sqrt{\det A}}}\sgn(\mu).
\]
The injectivity holds, and, if we know $A$, we can find the image. If $A>0$, then necessarily $\alpha>0$ and $\mu\in (0,+\infty)$.


\textbf{Case $d=2$, $k=2$}

We will take two matrices,
\[
    A_1=\begin{pmatrix}
        a_1&b_1\\b_1&c_1
    \end{pmatrix},\quad
    A_2=\begin{pmatrix}
        a_2&b_2\\b_2&c_2
    \end{pmatrix},
\] 
it will give us
\[Q=\alpha_1 A_1+\alpha_2 A_2,\]
\[\det Q = (\alpha_1 a_1+\alpha_2 a_2)(\alpha_1 c_1+\alpha_2 c_2)-(\alpha_1 b_1+\alpha_2 b_2)^2\]
\[  =  \alpha_1^2 a_1  c_1+\alpha_1\alpha_2 a_2 c_1+\alpha_1 \alpha_2 a_1 c_2+\alpha_2^2 a_2 c_2  - (\alpha_1 b_1)^2  -2\alpha_1\alpha_2 b_2 b_1 - (\alpha_2 b_2)^2 \]
\[
    =\alpha_1^2\det A_1 +2 \alpha_1 \alpha_2 A_1:co(A_2) + \alpha_2^2\det A_2
\]
Also
\[
    Q^{-1}=\frac{1}{\det Q}
    \begin{pmatrix}
        \alpha_1 c_1+\alpha_2 c_2&-(\alpha_1 b_1+\alpha_2 b_2)\\
        -(\alpha_1 b_1+\alpha_2 b_2)&\alpha_1 a_1+\alpha_2 a_2
    \end{pmatrix}
    =
    \frac{1}{\det Q}\left( \alpha_1 co(A_1)+ \alpha_2 co(A_2)\right)
\]
Thus
\[
    \mu_1 = \frac{A_1:Q^{-1}}{\sqrt{\det Q}}=\frac{1}{(\det Q)^{3/2}}\left( 2\alpha_1 \det A_1+ \alpha_2 A_1:co(A_2)\right)
\]
\[
    \mu_2 =\frac{A_2:Q^{-1}}{\sqrt{\det Q}}=\frac{1}{(\det Q)^{3/2}}\left(\alpha_1 A_2:co(A_1) + 2\alpha_2 \det A_2  \right)
\]

\end{document}
