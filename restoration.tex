
The problem we consider has the following formulation. Let

\[
    H = \begin{pmatrix}
        aI&bI-B\\bI+B&cI
    \end{pmatrix}
\]
with the condition that $H=H^T>0$. This condition is equivalent to the following identities:

\[
    a>0,\quad c>0,\quad B=-B^T,\quad Q=(ac-b^2)I+B^2=Q^T>0.
\]
Let the matrix $B=-B^T = \begin{pmatrix}
    0& -\gamma&  \beta\\
     \gamma&0&+\alpha\\
    -\beta& \alpha&0
\end{pmatrix}$, $\omega = (\alpha,\beta,\gamma)^T$.
Then we have
\[
    B^2
    =\omega \omega^T-|\omega|^2I
    = \begin{pmatrix}
    -\beta^2- \gamma^2& \alpha \beta& \alpha \gamma\\
    \alpha \beta& -\alpha^2- \gamma^2&\beta\gamma\\
    \alpha \gamma&\beta\gamma& -\alpha^2- \beta^2
    \end{pmatrix},
\]
eigenvalues of $B$ are $0$  and $\pm i|\omega|$. %with eigenvectors $\omega$  with eigenvectors
% \[find'em!\]
In these notations the matrix $Q$ becomes
\[
    Q=(ac-b^2-|\omega|^2)I+\omega \omega^T ,
\]
its eigenvalues are 
\[
    ac-b^2,\quad ac-b^2-|\omega|^2,\quad ac-b^2-|\omega|^2.
\]
Hence,
\[
    \det H =\det Q
    =(ac-b^2)(ac-b^2-|\omega|^2)^2.
\]
Also,
\[
    H^{-1}=\begin{pmatrix}
        cQ^{-1}&-(bI-B)Q^{-1}\\-(bI+B)Q^{-1}&aQ^{-1} .
    \end{pmatrix}
\]


We have the  following test values:

\[
    |x|^2\mapsto  \mu_c=\frac{c\tr (Q^{-1}) }{ \sqrt {\det(Q)}},
\]
\[
    |v|^2\mapsto  \mu_a = \frac{a\tr (Q^{-1}) }{ \sqrt {\det(Q)}},
\]
\[
    x\cdot v\mapsto  \mu_b=\frac{ \tr (-(bI+B)Q^{-1}) }{ \sqrt {\det(Q)}},
\]
\[
    x_2 v_3 - x_3 v_2\mapsto \mu_{\alpha} =\frac{   (-(bI+B)Q^{-1})_{23} -(-(bI+B)Q^{-1})_{32}  }{ \sqrt {\det(Q)}},
\]
\[
    x_3 v_1 - x_1 v_3\mapsto \mu_{\beta} =\frac{   (-(bI+B)Q^{-1})_{31} -(-(bI+B)Q^{-1})_{13}  }{ \sqrt {\det(Q)}},
\]
\[
    x_1 v_2 - x_2 v_1\mapsto \mu_{\gamma} =\frac{   (-(bI+B)Q^{-1})_{12} -(-(bI+B)Q^{-1})_{21}  }{ \sqrt {\det(Q)}}.
\]

First, we find the inverse of $Q$:
\[
    Q^{-1}=\frac{1}{ac-b^2-|\omega|^2}\left( I -\frac{1}{ac-b^2}\omega \omega^T \right),
\]
which leads instantly to
\[
    \tr(Q^{-1}) = \frac{1}{ac-b^2-|\omega|^2}\left(3-\frac{|\omega|^2}{ac-b^2}\right).
\]
In other words,
\[
    \mu_a =   \frac{ a\left(3-\frac{|\omega|^2}{ac-b^2}\right)}
    {(ac-b^2-|\omega|^2)\sqrt {\det(Q)}},
\]
\[
    \mu_c =  \frac{ c\left(3-\frac{|\omega|^2}{ac-b^2}\right)}
    {(ac-b^2-|\omega|^2)\sqrt {\det(Q)}},
\]
Next, since $B \omega =0$ and, of course, $B \omega \omega^T=0$, we can say that
\[
    (bI+B)Q^{-1} 
    =\frac{1}{ac-b^2-|\omega|^2} (bI+B) \left( I -\frac{1}{ac-b^2}\omega \omega^T \right)
\]
\[
    =\frac{1}{ac-b^2-|\omega|^2} \left( bI+B - \frac{b}{ac-b^2}\omega \omega^T \right),
\]
therefore
\[
    \tr (-(bI+B)Q^{-1}) 
    = -\tr \left( \frac{1}{ac-b^2-|\omega|^2} \left( bI+B - \frac{b}{ac-b^2}\omega \omega^T \right)\right)
\]
\[ 
    = \frac{1}{ac-b^2-|\omega|^2} \left( 3b   - \frac{b|\omega|^2}{ac-b^2}  \right)
\]
and hence
\[
    \mu_b = \frac{ b\left(3-\frac{|\omega|^2}{ac-b^2}\right)}
    {(ac-b^2-|\omega|^2)\sqrt {\det(Q)}}.
\]
Moreover, we can write
\[
    (-(bI+B)Q^{-1})_{12} -(-(bI+B)Q^{-1})_{21} 
\]
\[
    = - \frac{1}{ac-b^2-|\omega|^2} \left(  \left(  bI+B - \frac{b}{ac-b^2}\omega \omega^T \right)_{12} -\left(  bI+B - \frac{b}{ac-b^2}\omega \omega^T \right)_{21}   \right)
\]
\[
    =- \frac{1}{ac-b^2-|\omega|^2} \left(  B _{12} -  B_{21}   \right)
    = \frac{2 \gamma}{ac-b^2-|\omega|^2},
\]
which implies that
\[
    \mu_{\gamma} =  \frac{2 \gamma}{(ac-b^2-|\omega|^2)\sqrt {\det(Q)}}.
\]

For the simplicity of notations we will use the following vectors:
\[
    p = \begin{pmatrix}
        a\\b\\c
    \end{pmatrix},
    \quad
    l = \begin{pmatrix}
        \mu_a\\\mu_b\\\mu_c
    \end{pmatrix},
    \quad 
    g = \begin{pmatrix}
        \mu_{\alpha}\\\mu_{\beta}\\\mu_{\gamma}
    \end{pmatrix}.
\]
These vectors allow us to rewrite the above identities as
\[
    l = \frac{  \left(3-\frac{|\omega|^2}{ac-b^2}\right)}
    {(ac-b^2-|\omega|^2)\sqrt {\det(Q)}}p,
\]
\[
    g =  \frac{2 }{(ac-b^2-|\omega|^2)\sqrt {\det(Q)}}\omega.
\]
Let also $D_p=ac-b^2>|\omega|^2$, our identities become
\[
    l = \frac{  \left(3-\frac{|\omega|^2}{D_p}\right)}{(D_p-|\omega|^2)^2\sqrt { D_p}}p,
\]
\[
    g =  \frac{2 }{(D_p-|\omega|^2)^2\sqrt { D_p}}\omega.
\]
Let also 
\[
    D_l = \mu_a \mu_c- \mu_b^2,
\]
\[
    r = \frac{|\omega|^2}{D_p},\quad  r\in[0,1) .
\]
With these definitions we can say that
\[
    l = \frac{ (3-r )}{D_p^{5/2}(1-r)^2}p,
    \quad g = \frac{2 }{D_p^{5/2}(1-r)^2}\omega.
\]
This leads to
\[
    D_l = \left( \frac{ (3-r )} {D_p^{5/2}(1-r)^2}\right)^2 D_p = 
       \frac{ (3-r )^2} {D_p^{4}(1-r)^4}, 
\]
\[
    |g|^2 = \frac{4 }{D_p^{5 }(1-r)^4}|\omega|^2 =
    \frac{4 }{D_p^{4 }(1-r)^4}r,
\]
therefore,
\[
    \frac{|g|^2}{D_l}=\frac{4r}{(3-r )^2}.
\]
The map $r\to\frac{4r}{(3-r )^2} $ is strictly increasing and continuous on $[0,1)$ (with values in $[0,1)$), hence the condition $ \frac{|g|^2}{D_l}\in[0,1)$. If $z$ stands for $\frac{|g|^2}{D_l}$, then the inverse map writes
\[
    r(z)=3 - \frac{6}{1+\sqrt{1+3z}}.
\]
Afterwards, it becomes easy to deduce that
\[
    D_p = \frac{\sqrt{3-r}}{(1-r)\sqrt[4]{D_l}},
\]
and, finally,
\[
    p = \frac{D_p^{5/2}(1-r)^2}{ (3-r )}l,
\]

\[
    \omega = \frac{D_p^{5/2}(1-r)^2}{2 }g.
\]


\textbf{Remark:} we need to study this in terms of what we really obtain, since the matrix of moments is multiplied by a certain factor depending~--- and that brings trouble~--- by the determinant of this matrix. We also need to study the homogenity of our restoration process; if it is not invariant with respect to linear stretch, we have some troubles.

\textbf{Remark 2:} we need to look at the condition we obtain when we study the mass and several moments of the second order.

\textbf{NB!} This condition must be reducible to already known relations, otherwise, apparently, there is an error.

\textbf{note}
For a number $a$ and a matrix $H$ we obtain a number $m_0=\frac{f_0(a)}{\sqrt{\det H}}$ and a matrix $m_2 = \frac{f_2(a)}{\sqrt{\det H}}H^{-1}$.

This would lead us to a relation of the form
\[
        \det m_2  = f_2^{d}(a)(\det H)^{-\frac{d+2}{2}}, (\det H)^{-1/2} = \frac{m_0}{f_0(a)}.
\]
thus
\[
        f_2^{d}(a) = \det m_2 (\det H)^{\frac{d+2}{2}} = \det m_2 \left( \frac{f_0(a)}{m_0} \right)^{d+2}
\]
or
\[
        \frac{f_0(a)}{f_2^{\frac{d}{d+2}}(a)} = \frac{m_0}{(\det m_2)^{\frac{1}{d+2}}}.
\]
This is the origin % of fire =)
of the powers that we encountered. 

Other powers come from the integral representation of $f_0$ and $f_2$. We can write them (up to a constant mutliplier depending only on $d$) as

\begin{equation*}
        f_0(a) = \int_{0}^{+\infty} \bar f(r^2-a)r^{d-1}\dr, \quad f_2(a) = \int_{0}^{+\infty} \bar f(r^2-a)r^{d+1}\dr,
\end{equation*}
hence
\begin{equation*}
        \der{a} f_2(a) = -\int_{0}^{+\infty} \bar f'(r^2-a)r^{d+1}\dr = -\int_{0}^{+\infty}\frac{1}{2} \der{r} f(r^2-a)r^{d}\dr =  
\end{equation*}
\begin{equation*}
        =-\frac 12 \bar f(r^2-a)r^{d+1}\big|_0^{+\infty} + \frac d2 \int_{0}^{+\infty}f(r^2-a)r^{d-1}\dr = \frac d2 f_0(a).
\end{equation*}
This gives us the relation
\begin{equation*}
        \frac{f_0(a)}{f_2^{\frac{d}{d+2}}(a)} =  \frac{d+2}{2}\der{a} \left( f_2^{\frac{2}{d+2}}(a) \right)
\end{equation*}

